\documentclass[12pt, russian]{article}
\usepackage{cite}
\usepackage{algorithm}
\usepackage{algpseudocode}
\usepackage{amsfonts, amsmath, amssymb, amsthm, mathtools}
\usepackage[T2A]{fontenc}
\usepackage[utf8]{inputenc}
\usepackage{textcomp, cmap, comment}
\usepackage{graphicx, wrapfig}
\usepackage{epigraph, color, hyperref}
\usepackage{array, tabularx, tabulary, booktabs}
\usepackage{euscript, mathrsfs}

\newtheorem{theorem}{Theorem}

\begin{document}
    \begin{center}
        Дискретная математика и математическая кибернетика.
    \end{center}
    
    \section{Математическое программирование}
    \subsection{Теоремы о достижении нижней грани функции (функционала) на множестве (в ЕN, в метрических пространствах, в гильбертовых пространствах)}.
        % \begin{comment} Поляк Б.Т. Введение в оптимизацию. М.: Наука, 1984. \end{comment}
        
        \begin{theorem}[Первая теорема Вейерштрасса] Если функция $f(x)$ определена и непрерывна в замкнутом промежутке $[a,b]$ то она ограничена, т.е. существуют такие постоянные и конечные числа $m$ и $M$, что
        \begin{equation}
            m < f(x) < M \text{при } a < x < b
        \end{equation}
        \end{theorem}
        \begin{proof}
            Доказательство см. "Курс дифференционального и интегрального исчисления" под ред. Г.М. Фихтенгольца, Том 1 (стр. 175).
        \end{proof}
        
        \begin{theorem}[Вторая теорема Вейерштрасса] Если функция $f(x)$ определена и непрерывна в замкнутом промежутке $[a,b]$, то она достигает в этом промежутке своих точных верхней и нижней границ.
        \end{theorem}
        \begin{proof}
            Доказательство см. "Курс дифференционального и интегрального исчисления" под ред. Г.М. Фихтенгольца, Том 1 (стр. 176).
        \end{proof}
    
\end{document}